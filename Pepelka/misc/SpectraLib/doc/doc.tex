\documentclass[10pt]{article}
\setlength{\textheight}{9in} \setlength{\textwidth}{6.8in}
\setlength{\topmargin}{-0.6in} \setlength{\oddsidemargin}{-0.2in}
\begin{document}

\title{Spectral Clustering Toolbox}
\author{Deepak Verma ~~~~~~~~~~~~~~~~~~~~~~~~~~~~ Marina Meila \\\texttt{deepak@cs.washington.edu~~~~~~mmp@stat.washington.edu}}
\maketitle


\section{Introduction}
\label{sec:intro}

This toolbox contains the code written to perform various spectral
clustering algorithms. The details related to the code and some
experiments is available in \cite{VM03}. This document is very short and the
reader is encourage to look at the directories to see other code.  

\section{Using the toolbox}
\subsection{Quick Start}
\label{sec:quick}


To get up and cranking : 

\begin{enumerate}
\item  Set \textsc{SPECTRAL\_HOME} to be the directory where you unpacked the
  library.  
\item  Start matlab.  
\item  Call \texttt{init\_spectral}. (sets up the path
  and global options).  
\item 
  \texttt{assignment=cluster\_algo(similarity,number\_of\_clusters)} :
  Gives you the  desired clustering.     
\item  Remember that all the vectors that you see
  would be column vectors.
\end{enumerate}



\subsection{Data Input/Output}
\label{sec:dataio}


Reading a data file (see \texttt{data} directory for some examples). : 

\texttt{[similarity,cluster\_assignments,points]=read\_from\_data\_file(filePrefix,directory)}
\\
Reads the data file \texttt{directory/filePrefix} (default
dir=\texttt{data}) and assigns the \texttt{similarity}, the points and true
\texttt{cluster\_assignments}. If either of the the above is not defined empty
matrix is returned.




\subsection{Spectral Algorithms}
\label{sec:algos}


The algorithms are present in the \texttt{algos} and
\texttt{algos/allalgos} directory. The latter just contains files which
act convenient shortcut names to popular algorithms. Algorithm
\texttt{njw} is described in \cite{NgJW01} and \texttt{mcut} is
described in \cite{MeilaS00}. For the details and comparison of all the
algorithms see \cite{VM03}. 


\section{Experimental Framework}
\subsection{Running Experiments}
\label{sec:exp}

To run a bunch of experiments together use: 

\texttt{run\_single\_experiment(dataFile,cluster\_algo\_list,k\_range,sigma,iterations,outdir,plot\_points)}

This runs the experiments on \texttt{dataFile} for the algorithms
\texttt{cluster\_algo\_list} ,varying the input number of clusters in
the list \texttt{k\_range}. The \texttt{iterations} is the \emph{list}
of iterations indices and are useful when there is a random element in
the algorithm. \texttt{sigma} is the $\sigma$ used for affinity matrix (\cite{NgJW01}
in case the points (see section \ref{sec:dataio} are present in
\texttt{dataFile}. The results of each algorithm is written a file in
the \texttt{outdir} (with a default value used). If
\texttt{plot\_points} is 1 then the results are displayed after each
iteration for 2D points. (default 0). 


\subsection{Plotting graphs}
\label{sec:plot}

To the plot the graphs on the experiments ran using
\texttt{run\_single\_experiment} use:

\texttt{plot\_metric\_save(dataFile,cluster\_algo\_list,k\_range,iterations,metric,plot\_stdev,outdir)}

The arguments mean the same as above. \texttt{metric} is used  specify the metric to be used
to compare clustering produced w.r.t. true clustering. The metrics
available are 
\begin{itemize}
\item \texttt{vi} : Variation of Information (\cite{stat418}). 
\item \texttt{ce} : Clustering Error (see \cite{VM03} for details). 
  
\item \texttt{wi} : One sided Wallace Index (\cite{wallas}, also see
  \cite{VM03}). 
  
\end{itemize}



\section{Datasets}
\label{sec:data}


\subsection{Artificial}
\label{sec:artdata}

Some artificial datasets are provided in the \texttt{data}
directory. All of them are 2D points which offers various levels of
difficulty to the spectral algorithms. They are modelled after
\cite{NgJW01}. To see these (or any other 2D) plots use
\texttt{plot2Dpoints\_with\_clusters}. 

An interesting dataset (not in 2D) called \texttt{block-stochastic}
(\cite{MeilaS00}) is also provided. It is a similarity matrix designed
(\cite{VM03}) to illustrate the case when spectral methods work and
linkage based methods fail. 

\subsection{Real Datasets}
\label{sec:realdata}

Coming soon....

\section{Demo}

Run \texttt{spectral\_demo} in the \texttt{demo} directory for seeing typical
use of the library functions.   
\small{
  \bibliography{references}
  \bibliographystyle{alpha}
}

\end{document}
